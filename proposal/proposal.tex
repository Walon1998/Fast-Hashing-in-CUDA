\documentclass[a4paper]{article}
\usepackage[utf8]{inputenc}
\usepackage{forloop}

\newenvironment{reqlist}{%
  \par \medskip \noindent
  \begin{tabular}{cp{0.83\textwidth}r} \\[-24pt]}{\end{tabular}}
\newcommand{\req}{\\ \smallskip \smallskip $\bullet$\hspace{-0.2cm} & }
\newcounter{num}
\newcommand\effort[1]{\mbox{(\forloop{num}{0}{\value{num} < #1}{$\star$})}}

\author{Neville Walo, Basil F\"urer}
\title {DPHPC Project Proposal}

\begin{document}
\parindent 0pt
\maketitle

\section*{Fast Hashing with GPU}

Hashing finds many applications today and often we are interested in fast
computation of a lot of hashes. This is mostly due to the rise of
cryptocurrencies but it also has applications in generating rainbow tables.

Many cryptocurrencies are based on the notion of a \textit{proof of work}, a
process with a low probability of succeeding. For example, Bitcoin asks its
miners to do a proof of work that includes all of the data in the current block.
This process is based on the SHA-256 hashing function.~\cite{nakamoto2012bitcoin}
\\

The goal of this project is to accelerate hash computations by using a GPU.
SHA-256 \cite{Dang13} used in the Bitcoin protocol has a lot sequential
dependencies and a single hash computation is unlikely to be parallelizable,
however the computation of many different hashes in parallel is possible. Our
project consists of the following milestones:

\begin{reqlist}
  \req literature research, settling for a concrete hash function $H(\cdot)$
    & \effort{1}
  \req implement $H$ with CUDA~\cite{cuda}
    & \effort{3}
  \req evaluate and benchmark our implementation against existing
       implementations~\cite{bench}
    & \effort{2}
\end{reqlist}

\hfill

\bibliographystyle{unsrt}
\bibliography{bibl}

\end{document}
